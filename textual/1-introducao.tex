\section{Introdução}
\vspace*{1cm}
O objetivo central deste projeto é correlacionar pesquisas em prol das melhorias derivadas do uso de totens de auto-atendimento, tanto no tocante a celeridade no atendimento quanto a melhoria em acessibilidade dos serviços. Também discorre sobre as necessidades pesquisadas em CCM's (Centro de Cidadania da Mulher) e como essas centrais de auto-atendimento podem trazer melhoria nos serviços ofertados, maior abrangência em zonas próximas ao Centro mas que desconhecem sua existencia. A proposta final busca facilitar o acesso aos serviços do programa, medir sua abrangência e identificar necessidades de orçamento e/ou novos centros em zonas não atendidas.
\subsection{Motivação}
O objetivo dos CCMs é oferecer serviços e atendimentos à mulhers que passaram por violência doméstica e precisam tanto de atendimentos jurídico quanto acompanhamento psicológico e, mesmo com um número alto de casos de violência doméstica, carece de visibilidade nos serviços ofertados. Tendo sido implantados de 2005 a 2007 na cidade de São Paulo, os CCMs promovem eventos e oferecem serviços que estimulam mulheres a valorizar e garantir seus direitos (\cite{centros-de-cidadania-estimulam-mulheres-a-valorizar-seus-direitos}).
Por ser um programa público, tem orçamento oriundo de fundos da prefeitura de São Paulo (\cite{politicas-para-mulheres-tem-orcamento-reforcado-com-proposta-de-vereadoras}) e, como gerador de renda, oferecem também oficinas de artesanato, pintura, bijuteria, etc.

É nesse contexto que se insere a motivação deste projeto, a de criar um acesso fácil e ágil para que seja possível atingir mais mulheres afetadas e que seja possível medir a eficácia e visibilidade do programa de modo que seja possível, em futuras deliberações da câmara paulista, trazer maiores recursos ao orçamento destinado aos Centros.

\subsection{Localização}
De acordo com pesquisa realizada pela Companhia Paulista de Trens Metropolitanos (CPTM), realizada em março/2018, (\cite{cptm-demandas-publico-feminino}), as mulheres que, em pesquisa datada de 2016 representam 56\% do número de passageiros dos metrôs paulistas (\cite{pesquisa-de-caracterizacao-socioeconomica-do-usuario}), pediram por serviços que otimizem seu dia-a-dia. É diante dessa necessidade que a proposta de adicionar totens de atendimento ao público feminino em estações de transporte público como metrôs e trens se faz relevante, pois oferece chance às mulheres que já conhecem os serviços do CCM de acompanharem o andamento de eventuais processos jurídicos bem como, agendar novas atividades e agendar horários de acompanhamento psicológico e atinge mulheres que desconhecem o programa de modo que passem a usufruir do mesmo.
No que tange medição de efetividade, os totens podem fornecer informações a um registro geral dos CCM sendo possível identificar as estações onde existem mais mulheres carentes do serviço, metrificar o uso e necessidade de orçamento e eventual identificação da necessidade de se criar novos centros em zonas de maior adesão ao uso.
% Para iniciar um novo parágrafo pule uma linha
\newpage

