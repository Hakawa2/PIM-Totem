\section{Introdução}
\vspace*{1cm}
Primeiramente, tem se três termos que são necessários para uma melhor compreensão de um projeto na área de TI (Tecnologia da Informação) como o presente. São eles: Sistema, Transacional e Multiplataforma.\par
O primeiro seria basicamente um mecanismo, manual ou automatizado, que tem como finalidade coletar, processar, armazenar e transmitir dados, de forma a ser útil para a empresa pela qual foi desenvolvida, atendendo suas necessidades e solucionando seus problemas. Sistemas transacionais são essencialmente o sistema operacional de uma empresa, e assim como o próprio nome sugere está relacionada com as transações da mesma, atendendo normalmente à área administrativo-financeira. Por fim, multiplataforma representa o fato do sistema poder ser executado em mais de uma plataforma. Tendo como exemplo o presente trabalho, em que se busca projetar a ideia de FoF “For Freelancers” em três plataformas diferentes (desktop, web e mobile). Neste caso, a parte web e mobile serão bastante parecidas em algumas funções e tendo uma mesma base.\par
Antes mesmo de se entender o que é a FoF, precisamos introduzir o seu conceito e distribuir suas ideias para que se possa ter uma compreensão completa de como será seu funcionamento. \par
Entrando no tema escolhido, para início, precisa ser explicado o que é a sigla FoF. Para melhor entendimento, o significado dos termos em “For Freelancers” seriam respectivamente “para” e “profissionais autônomos”, ambos do inglês. No Brasil, talvez o significado do segundo termo pode não fazer tanto sentido, por aqui ser muito associado a bicos, serviços temporários. \par
Precisa-se lembrar que apesar de terem algumas semelhanças, os termos freelancer e bico não são sinônimos. O primeiro é associado a construção de uma carreira, mesmo que atue nos momentos livres de seu emprego principal, já o segundo não tem como foco a construção de uma carreira, mas sim como um trabalho temporário enquanto se está desempregado ou quer juntar dinheiro extra. Além disso, diferente do bico, o freelancer precisa de muito estudo e atualização, executando um trabalho especializado e formal.\par
A palavra Freelancer cresceu no Brasil graças ao “hype” (algo que está chamando muita atenção) das empresas novas de tecnologia que introduzem termos estrangeiros em suas unidades e acabam pegando na boca do povo. Com isso em mente decidimos criar a FoF, focada em atender as necessidades de freelancers na área de TI, e ajudar as empresas da mesma área a gerenciar o fluxo dos freelancers de uma maneira mais fluida e fácil.
\subsection{Objetivo}
Como dito na introdução, nosso foco é gerenciar e automatizar esse mercado de freelancers
que tanto cresce. \par
Tendo como objetivo central, na parte empresarial, desenvolver uma central criada de forma
categorial para auxílio, e na parte do freelancer, uma central de empresas na qual seu perfil
mais se identifica. Nos seguintes subtópicos, tem-se uma explicação um pouco mais especifica,
de forma a mencionar os objetivos para a presente ideia em cada uma das plataformas,
desktop, web e mobile.

\subsubsection{Plataforma Desktop}
O objetivo da aplicação console é expandir ainda mais o mercado, hoje
se tem muitas empresas que gerenciam as agendas e seus locais de serviços de diaristas, pensando nisso a ideia de criar uma aplicação voltada para empresas poderem
divulgar seus bancos de freelancers de maneira rapida e prática surgiu.

\subsubsection{Plataforma Web}
No site você pode realizar todo o cadastro próprio (funcionário/freelancer) ou
cadastro da empresa, visando a facilidade de escrever sobre a empresa ou funcionário, podendo também realizar manutenções nos perfis, e ter uma busca mais avançada
através do site.

\subsubsection{Plataforma Mobile}

Com a versão mobile nosso sistema irá propor uma praticidade de acesso ao
público e para as pessoas cadastradas na FoF, de forma que possam efetuar e acompanhar seu perfil e processos cadastrais. Na plataforma mobile você poderá realizar
seus cadastros de maneira segura e com grande facilidade, podendo preencher ficha
curricular e anexar currículos, além de fazer alterações no seu perfil, quando necessário.

\subsection{Justificativa}

O mercado de TI no Brasil anda crescendo de maneira incrível, cada dia que passa se surge
uma empresa nova, ou até mesmo uma startup (empresas pequenas que surgem, e buscam
evoluir quase que diariamente, mesmo que talvez não se tenha o projeto completo e sim um
pedaço de um protótipo). Com isso a busca por profissionais da área cresceu demais.

O que mais alavancou as buscas é que uma startup da noite para o dia pode conseguir um
grande investimento e necessitar transformar aquele protótipo em um projeto completo, e, ao
não se ter tempo de contratar pela CLT(Consolidação das Leis do Trabalho), essas startup ou
empresas acabam buscando freelancers.

Então, se tem um mercado de um lado crescendo que são as empresas, do outro a busca por
freelancers cresce paralelamente. Apesar disso, não se tem alguém para gerir e fazer uma
gestão de modo que ajude os dois lados da moeda. Vendo essa necessidade crescendo vem a
FoF para desafogar os RH que precisam ficar navegando horas e horas atrás de profissionais.

\subsubsection{Plataforma Desktop}

Hoje se tem muitas empresas que gerenciam as agendas e seus locais de
serviços de diaristas, pensando nisso a ideia de criar uma aplicação voltada para
empresas poderem divulgar seus bancos de freelancers de maneira rápida e prática
surgiu.

\subsubsection{Plataforma Web}

Com a versão do nosso software especificamente para web, facilitaremos o
acesso as pessoas com a possibilidade de acessar na nossa plataforma de qualquer
lugar através de basicamente qualquer aparelho que tenha acesso a internet fazendo
assim com que o nosso software seja algo extremamente versátil e prático de ser
usado

\subsubsection{Plataforma Mobile}

Um dos pontos chave é que nossa parte mobile, assim como a parte web, preza
ao máximo a UX (user experience) e a UI (user interface), para que nossos clientes
se sintam a vontade para realizar e modificar seu cadastro em qualquer plataforma,
tendo a melhor experiência de usabilidade. Observando que a UX e UI estão em alta
no mercado, pois as empresas estão buscando centralizar seus layouts em todas as
plataformas.
% Para iniciar um novo parágrafo pule uma linha
\newpage

