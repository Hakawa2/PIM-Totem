\section{Referencial Teórico}
\vspace*{1cm}
Esse capítulo, a título de embasamento teórico dessa pesquisa, trata de uma revisão de literatura acerca da eficiência e justificativa para o uso de tecnologias de autoatendimento, comportamento do usuário e aceitação da tecnologia, bem como busca estudos sobre edicácia de serviços públicos de atendimento à mulher que tenha passado por situação de violência doméstica.
\subsection{Justificativa do uso de autoatendimento}
A introdução das tecnologias de autoatendimento pelas empresas tem o objetivo de aumentar a produtividade e eficiência, além de oferecer aos consumidores novos canais de acesso aos serviços, para melhor atenderem suas demandas (\cite{autoatendimento-em-aeroportos}; \cite{walker-service-delivery}).

Do ponto de vista de Bitner, a tecnologia pode aumentar drasticamente o número de relações entre consumidor e empresa, visto que, ao invés de esperar um comunicado a períodos de tempo pré-estabelecidos, o consumidor pode consultar seus dados de modo autônomo (\cite{bitner-service-encounters}).
\subsection{Aceitação de tecnologias por parte do público}
Dabholkar levantou em seus estudos, em 1996, uma relação entre a preferência por autoatendimento e a percepção do agradável, conveniente (\cite{dabholkar-service-encounters}).

Enquanto Sanders, notou que a percepção de facilidade no uso é menor quanto maior for a idade do usuário, ainda que a associação entre as variáveis seja fraca. Ainda no estudo de Sanders é possível visualizar uma maior tendência no uso e aceitação do autoatendimento às pessoas com maior grau de escolaridade (\cite{autoatendimento-em-aeroportos}).
\subsection{Serviços públicos}

\newpage

