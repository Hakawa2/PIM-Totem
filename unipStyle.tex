\usepackage[english,portuguese]{babel} %linguagem do documento
\usepackage[fixlanguage]{babelbib}
\selectbiblanguage{portuguese}
\usepackage[colon,authoryear]{natbib}
\usepackage[T1]{fontenc} 
\renewcommand\familydefault{\sfdefault}
\usepackage{uarial}
\usepackage[utf8]{inputenc} %cod. para o Sistema Operacional
\usepackage{inslrmin}
\usepackage{marvosym}%simbolos diversos 
\usepackage[pdftex]{graphicx}
\usepackage{pdfpages} %permite inserir .pdf
\usepackage{rotating}
\usepackage{booktabs} %Tabelas com qualidade de publicação 
\usepackage[hyphens]{url}
\usepackage{amssymb, amsmath, pxfonts, amsfonts, textcomp} %simbolos matematicos
\usepackage{mathrsfs} %uso de fontes para conjuntos
\usepackage{wasysym} % para inserção de simbolos(quadradinhos, bolinhas, etc)
\usepackage[normalem]{ulem} %sublinhar palavras
\newenvironment{remark}{\par\mbox{}\par\noindent{\bfseries \emph{Remark}\/}:\normalfont}{\par}
\newenvironment{defin}{\par\mbox{}\par\noindent\bfseries \emph{\Pointinghand\ }\normalfont}{\par}
\usepackage[colorlinks=true, a4paper=true, pdfstartview=FitV,
linkcolor=blue, citecolor=blue, urlcolor=blue, breaklinks=true]{hyperref}
\pdfcompresslevel=9
\usepackage{indentfirst}%identação do primeiro parágrafo
\usepackage{setspace}%permite espaço entrelinhas
\onehalfspacing %definição do espaço entre linhas de 1.5
\usepackage[left=3cm, right=2cm, bottom=2cm, top=3cm]{geometry}%margens
\usepackage{fancyhdr}%notas de rodapé
\usepackage[small]{caption} %Customizar as legendas de figuras e tabelas
\renewcommand{\thefigure}{\arabic{figure}}
\renewcommand{\thetable}{\arabic{table}}
\usepackage{listings}%para a inserção de códigos Java, C++, etc.
\usepackage{xcolor}%para redefinição de cores 
\usepackage{color} %permite letras coloridas 
\usepackage{epic}%para plotar gráficos
\usepackage{float} %flutuação de figuras e tabelas, utilize com [H] para fixar.
\parindent=1.25cm 
\usepackage{multirow}
\usepackage{lscape}
\usepackage[export]{adjustbox}
\pagestyle{myheadings}
\fancyhead[H]{}
%\topmargin=0pt


